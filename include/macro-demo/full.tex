\documentclass[12pt,letterpaper]{article}
\usepackage{fretplot}
\usepackage{float} % For [H] placement specifier in figures.

\begin{document}

\begin{figure}[H]
\fpscale{parentroot=C#|formulatype=d|formula=1 2 b3 4 5 b6 7}
\caption{C$\sharp$ harmonic minor scale}
\end{figure}

% Create a fretplot file include/sclinclude.fp
% and a fretplot scale style file include/styles.fps.
% Comment these 2 lines out after the first run if
% you do not want to overwrite them in subsequent runs.
\fptemplate{include/sclinclude.fp}
\fpstemplate{include/styles.fps}

% Use fptotikz to test include/sclinclude.fp
% generated by \fptemplate.
\begin{figure}[H]
\fptotikz{include/sclinclude.fp}
\caption{Test \texttt{include/sclinclude.fp} generated by \texttt{\textbackslash fptemplate}}
\end{figure}

% Also test the TikZ file generation capability of
% \fptotikz, although the rest of the code does not
% depend on this. Note that this will merely create
% the file tikz/sclinclude.tex containing the
% TikZ code, but not include it in the document.
\fptotikz[tikz/sclinclude.tex]{include/sclinclude.fp}

\begin{figure}[H]
\fpscale{parentroot=A|formulatype=i|formula=\fpmaj|mode=2|
tuning=E B G D A E|numfrets=12|styletype=d|
labeltype=p|scalestylefile=include/styles.fps|
includefpfile=include/sclinclude.fp}
\caption{2\textsuperscript{nd} mode of A major scale (B Dorian)}
\end{figure}

\end{document}
